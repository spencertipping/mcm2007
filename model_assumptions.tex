% vim: set tw=80 ts=4 sw=4 :

\documentclass{article}
\usepackage{palatino}

\begin{document}
\section{The Graph Model}
\subsection{Introduction}

The graph model builds the airplane seating out of a graph of nodes, each
connected bidirectionally to applicable adjacent nodes in one of the four
cardinal directions. Each node contains an occupant. Aisles have connections in
all directions and seats have connections horizontally.

Passengers are all tracked exactly as they pass through the plane. Each one of
them is randomly assigned with uniform probability zero, one, or two carry-on
bags that must be stored in an overhead bin before the passenger is seated. It
is assumed that the passenger will take the aisle closest to the assigned seat
in every case, crossing no more seats than is absolutely necessary.

Overhead bins are considered to be shared among several rows, usually two or
three. The time required to load one additional bag into an overhead bin is
directly proportional to the square root of the number of bags currently there.
Thus, loading bags is of 3/2 order for the number of bags to load.

Because of the structural flexibility present in the model, it is able to
emulate planes with non-consistent geometries, such as the Boeing 767-400 with
2-2-2 in the front and 2-3-2 in the back. We can also, although with more
difficulty, implement planes with two floors such as the Airbus 380.

The graph model is able to use a smaller sample size because of the
recomputation of random data. Every time a node's delay is computed, it is
re-randomized; thus, one single run will incorporate a broadly normalized set of
random data. For this reason, we consider a sample size of 200 runs per
configuration to be sufficient to accurately represent the performance of the
different configurations.

\subsection{Parameters}
The major parameters of this model are:

\begin{description}
	\item[Aisle-aisle movement delay]
		This determines how long it takes for a person to move one node through
		an aisle. By default, the value is two seconds.

	\item[Aisle-seat movement delay]
		How long it takes to move from an aisle to a seat. By default, this is
		three seconds.

	\item[Seat-aisle movement delay]
		How long it takes to move from a seat to an aisle. By default, this is
		3.5 seconds.

	\item[Seat-seat movement delay]
		How long it takes to move from one seat to another. By default, this is
		seven seconds.

\end{description}

\subsection{Strengths}

One strength includes accurate simulation of shared luggage bins: A passenger
loading a bag into a bin two rows ahead may influence the loading time of a
piece of baggage elsewhere. In addition, luggage bins are shared for both sides
of an aisle, which accurately models people's tendency to put luggage on either
side of the aisle.

Another is that aisle spaces are allocated for people moving across already
taken seats, simulating the requirement of everyone clearing the occupied seats
for the newcomer to move in. This accentuates the effectiveness of modifications
to the strategies such as the even-odd variation or the staggered variation.

A third strength is that if there is an aisle, an empty seat, a filled seat, and
the target seat, in that order, and a passenger moves into the empty seat on his
way to the target seat, he can get to the target seat only when the aisle is
clear. This is by the rationale that all swapping must be done through an aisle.

\subsection{Weaknesses}

One weakness of the model is that it does not simulate people traveling very far
to get to an empty luggage bin. Rather, luggage bins are assumed to have
infinite capacity (although they would incur very large delays), and people do
not ``prefer'' those bins with smaller delays. Also, the bin on the east side of
the aisle is not distinct from the one on the west. This may affect the model's
accuracy slightly.

Further, passengers, when filling the aisle to make room for someone who needs
access to a window seat, do not ever look north; if the south cell cannot be
taken, then they incur extra time delays. This is somewhat inaccurate.

\end{document}
